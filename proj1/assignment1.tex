\documentclass[fleqn,10pt,titlepage]{article}

\usepackage{graphicx}                                        

\usepackage{amssymb}                                         
\usepackage{amsmath}                                         
\usepackage{amsthm}                                          

\usepackage{alltt}                                           
\usepackage{float}
\usepackage{color}

\usepackage{url}

\usepackage{balance}
\usepackage[TABBOTCAP, tight]{subfigure}
\usepackage{enumitem}

\usepackage[top=0.8in, left=1in, bottom=0.8in, right=1in]{geometry}

\usepackage{pstricks, pst-node}

\usepackage{geometry}
\geometry{textheight=10in, textwidth=7.5in}

\newcommand\numberthis{\addtocounter{equation}{1}\tag{\theequation}}
\newcommand{\cred}[1]{{\color{red}#1}}
\newcommand{\cblue}[1]{{\color{blue}#1}}

\usepackage{hyperref}

\def\name{Dean Johnson}

%% The following metadata will show up in the PDF properties
\hypersetup{
  colorlinks = true,
  urlcolor = black,
  pdfauthor = {\name},
  pdfkeywords = {cs311 ``operating systems'' files filesystem I/O},
  pdftitle = {CS 311 Project 1: UNIX File I/O},
  pdfsubject = {CS 311 Project 1},
  pdfpagemode = UseNone
}

\pagenumbering{gobble}
\numberwithin{equation}{section}
\parindent = 0.0 in
\parskip = 0.2 in

\begin{document}

\section{Part 1 -- 6 questions}
\subsection*{1. Describe at least 2 ways of transferring files from a remote server to a local machine.}

scp {[host:filepath on host]} {[destination]} ** Add the -r argument for more than 1 file or a directory. \\
rsync {[host:filepath on host]} {[destination]} ** Add the -r argument for more than 1 file or a directory.

\subsection*{2. What are revision control systems? Why are they useful? Explain how to create a subversion or git repository on os-class.}

A revision control system is a tool used to track, and save, progress on a project. They are useful for 
tracking documents that multiple people are working on in sync, and up-to-date. They are usually by programmers, to keep parts of a 
project isolated from eachother, so issues with one portion of the project aren't conflicting with other portions of the project. 
To create a git repository on os-class, run: \\
git init {[repository name]}

\subsection*{3. What is the difference between redirecting and piping? Describe each.}
Piping is used to pass output from a program to another program, whereas redirecting is used for passing output from something other%
that a program to a file or input stream.

\subsection*{4. What is make, and how is it useful?}
Make is a utility in linux, used for executing shell commands. It is usually used to compile files.

\subsection*{5. Describe, in detail, the syntax of a makefile.}
The syntax for a makefile goes: \\
{[Dependencies]} (These include the the target and source for the compilation) \\
{[Actions/Commands]} (These include the commands to run to compile the code)

\subsection*{6. Give a find command that will run the file command on every regular file (not directories!) within the current filesystem subtree.}
find -type f -exec file \{\} \textbackslash ;

\end{document}
