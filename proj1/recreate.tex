\documentclass[fleqn,10pt,titlepage]{article}

\usepackage{graphicx}                                        

\usepackage{amssymb}                                         
\usepackage{amsmath}                                         
\usepackage{amsthm}                                          

\usepackage{alltt}                                           
\usepackage{float}
\usepackage{color}

\usepackage{url}

\usepackage{balance}
\usepackage[TABBOTCAP, tight]{subfigure}
\usepackage{enumitem}

\usepackage[top=0.8in, left=1in, bottom=0.8in, right=1in]{geometry}

\usepackage{pstricks, pst-node}

\usepackage{geometry}
\geometry{textheight=10in, textwidth=7.5in}

\newcommand{\cred}[1]{{\color{red}#1}}
\newcommand{\cblue}[1]{{\color{blue}#1}}

\usepackage{esint}
\usepackage{hyperref}
\usepackage{wasysym}

\def\name{Dean Johnson}

%% The following metadata will show up in the PDF properties
\hypersetup{
  colorlinks = true,
  urlcolor = black,
  pdfauthor = {\name},
  pdfkeywords = {cs311 ``operating systems'' files filesystem I/O},
  pdftitle = {CS 311 Project 1: UNIX File I/O},
  pdfsubject = {CS 311 Project 1},
  pdfpagemode = UseNone
}

\pagenumbering{gobble}
\numberwithin{equation}{section}
\parindent = 0.0 in
\parskip = 0.2 in
\setcounter{section}{1}

\begin{document}
\section*{Surface Integral}
\hrule
For a scalar function \textit{f} over a surface parameterized by \textit{u} and \textit{v}, the surface integral is given by

\begin{align}
    \Phi& = \int_{S} f \, \rm{d}\textit{a}\\
    & = \int_{S} f \,(\textit{u}, \textit{v}) |\textbf{T}_u \times \textbf{T}_v |\, \rm{d}\textit{u} \, \rm{d}\textit{v}
\end{align}

where \textbf{T}$_u$ and \textbf{T}$_v$ are tangent vectors and $a \times b$ is the corss product.\\ \\
For a vector function over a surface, the surface integral is given by

\begin{align}
    \Phi & = \int_{S} \textbf{F} \cdot \, \rm{d} \textbf{a} \\ 
    & = \int_{S}(\textbf{F} \cdot \hat{\textbf{n}})\,\, \rm{d} \textit{a} \\
    & = \int_{S} f_x \,\rm{d}\textit{y} \,\, \rm{d}\textit{z} \, + \, \textit{f}_\textit{y} \, \rm{d}\textit{z} \,\, 
    \rm{d}\textit{x} \, + \, \textit{f}_\textit{z} \, \rm{d}\textit{x} \,\, \rm{d}\textit{y}
\end{align}

where $\textbf{a} \cdot \textbf{b}$ is a dot product and $\hat{n}$ is a unit normal vector. If $z = f(x, y)$,
 the d\textbf{a} is given explicitly by

\begin{align}
    \rm{d}\textbf{a} \, = \, \pm \left(\frac{\partial\textit{z}}{\partial\textit{x}}\hat{x} \, - \, 
    \frac{\partial\textit{z}}{\partial\textit{z}}\hat{y}\, + \, \hat{z}\right) \, \, 
    \rm{d}\textit{x} \, \, \rm{d}\textit{y}
\end{align}

If the surface is \textit{surface parameterized} using $u$ and $v$, then

\begin{align}
    \Phi = \int_{S} \textbf{F} \cdot (\textbf{T}_u \times \textbf{T}_v )\, \rm{d}\textit{u} \, \, \rm{d}\textit{v}.
\end{align}

\clearpage

%\numberwithin{equation}{section}
\setcounter{section}{2}
\setcounter{equation}{0}
\section*{Maxwell's Equations}
\hrule
\begin{center}
\includegraphics[scale=0.44]{maxwell.eps}
\end{center}

{\large \textbf{Integral Form}}


\clearpage

\section*{Grading Policies}
\hrule

\begin{enumerate}
\item{All project must be submitted electronically by 23:59:59 on the due date via TEACH -- use "Check
time on server" if unsure about your clock. TEACH time takes priority over your local computer.}
\item {Only a single late homework assignment allowed. Only allowed up to 7 calendar days late.}
\item {Submit late homework to your assigned TA via email.}
\item {Blatant disrespect to or by the TAs will not be tolerated.}
\item {If you do not demo your project, you do not receive credit for it.}
\item {When you make an appointment to demo, show up. Failure to show up will result in a grade penalty.
Repeated offenses will result in no credit for the assignment.}
\item {If your project does not compile, for any reason, no credit will be.}
\item {Compilation will be on os-class. This server is the final say on whether your code compiles.}
\item {No directories in you submissions. You will be penalized for including any sort of hierarchy.}
\item {All assignments submitted to TEACH. No late submissions will be accepted via TEACH.}
\item {Naming convention: CS311 projhxi hengr usernamei.tar.bz2. Fill in hi with appropriate values.}
\item {No zip files will be accepted. You must use bzipped tar files.}
\item {All non-code documents must be created with LaTeX, by hand. This will be discussed in class.}
\item {All work must be done individually unless specifically allowed to work in groups.}
\end{enumerate}

\section*{Learning Objectives}
\hrule

\begin{itemize}
\item{Explain why multiprogramming is important for modern operating systems.}
\item{Explain the general structure of a multiprogrammed operating system.}
\item{Explain the purpose and operation of system calls.}
\item{Write a program utilizing system calls.}
\item{Write a program using a scripting language.}
\item{Write a program that uses regular expressions to parse input data.}
\item{Write a program that spawns processes and provides mutual exclusion for variables or other resources
shared by the processes.}
\item{Write a program that uses sockets to implement a client/server system.}
\item{Explain how a common file system works, including structure, I/O operations, and security.}
\item{Describe the memory organization of a typical process in a common operating system.}
\end{itemize}



%input the pygmentized output of mt19937ar.c, using a (hopefully) unique name
%this file only exists at compile time. Feel free to change that.
%\input{__mt.h.tex}
\end{document}
