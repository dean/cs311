\documentclass[fleqn,10pt,titlepage]{article}

\usepackage{balance}
\usepackage[TABBOTCAP, tight]{subfigure}
\usepackage{enumitem}

\usepackage[top=0.8in, left=1in, bottom=0.8in, right=1in]{geometry}

\usepackage{pstricks, pst-node}

\usepackage{geometry}
\geometry{textheight=10in, textwidth=7.5in}

\newcommand{\cred}[1]{{\color{red}#1}}
\newcommand{\cblue}[1]{{\color{blue}#1}}

\usepackage{longtable, hyperref}
\newcommand{\longtableendfoot}{Please continue at the next page}

\def\name{Dean Johnson}

%% The following metadata will show up in the PDF properties
\hypersetup{
  colorlinks = true,
  urlcolor = black,
  pdfauthor = {\name},
  pdfkeywords = {cs311 ``operating systems'' files filesystem I/O},
  pdftitle = {CS 311 Project 2: UNIX File I/O},
  pdfsubject = {CS 311 Project 2},
  pdfpagemode = UseNone
}

\pagenumbering{gobble}
\parindent = 0.0 in
\parskip = 0.2 in

\begin{document}

\section{Design}
\hrule
\subsection{Initial Implementation}
My overall idea for an inital design involves using the struct in the ar.h file, to store all information about the archive, and then 
information on each archived file below it, following the specifications listed in ar.h.
\subsubsection{Archiving}
First write information about the files, to the file descriptor of the archive. Store this first. Next, follow the same steps on each
file provided in the arguments, by saving the file descriptor information, then the content directly afterwards.

\subsubsection{Extraction}
Read the information about the archive file, then begin extraction on the first file stored inside, and so on.
\subsection{Deviations}
Originally I thought that the archive had it's own ar header, but then I realized it didn't. This made things much easier.
\clearpage

\section{Work Log}
Git log...

commit 15fae003cd9826be4e76cbae9f95af4c43b9140f
Author: Dean Johnson <deanjohnson222@gmail.com>
Date:   Wed Oct 30 23:09:07 2013 -0700

    Cleanup

commit 202b4a7e15adb791e34eb76c56b59b49d7cea565
Author: Dean Johnson <deanjohnson222@gmail.com>
Date:   Wed Oct 30 23:06:35 2013 -0700

    Finished -t, -v, -q

commit 93b524eccd0ef7ca93aabfc5db498db86343e5e2
Author: Dean Johnson <deanjohnson222@gmail.com>
Date:   Wed Oct 30 22:33:43 2013 -0700

    Added -t functionality and fixed -q.

commit 0c121c5eb041a5741f57297e5ef90542d53d7761
Author: Dean Johnson <deanjohnson222@gmail.com>
Date:   Wed Oct 30 19:53:41 2013 -0700

    Added -q functionality

commit 47defae5a19bd753281fe6c409a34d7df45fb916
Author: Dean Johnson <deanjohnson222@gmail.com>
Date:   Wed Oct 30 18:57:27 2013 -0700

    Added lots of functionality to -q

commit 79b38421987dd2af416a21bd753c5a5a801fc90c
Author: Dean Johnson <deanjohnson222@gmail.com>
Date:   Mon Oct 28 19:33:41 2013 -0700

    Added partial functionality for archive appending.
    
    Added the archive file header assignment, for archive appending.
    Still need to write the rest of the arguments.

commit 8c65caee0301a863d8adb0f3d9b1a7e0746dbfe1
Author: Dean Johnson <deanjohnson222@gmail.com>
Date:   Sat Oct 26 16:07:36 2013 -0700

    Outlined most of the assignment in the code.

commit fe42c7b54d42b5fe8bb7b99beab047d1f857b3ed
Author: Dean Johnson <deanjohnson222@gmail.com>
Date:   Fri Oct 18 13:44:04 2013 -0700

    Added initial design to TEX file.
    
    Also updated makefile.

commit 284c28746481c955e5692a929ace45fd3143384f
Author: Dean Johnson <deanjohnson222@gmail.com>
Date:   Fri Oct 18 13:20:49 2013 -0700

    Added writeup template.
    
    Also editted the makefil to compile both my TEX document and C
    code.

commit 03d5e4951b4728b3d78152bdcd642a5bdca198da
Author: Dean Johnson <deanjohnson222@gmail.com>
Date:   Thu Oct 17 20:31:18 2013 -0700

    Added makefile

commit c3a93a02617cab1c1db3efdae0f5487d0dc68e3f
Author: Dean Johnson <deanjohnson222@gmail.com>
Date:   Thu Oct 17 20:25:04 2013 -0700

    Starting second assignment

commit 78019014c9baa462f98e0a7c94c11ff13baef36a
Author: Dean Johnson <johnsdea@flip1.engr.oregonstate.edu>
Date:   Thu Oct 17 20:20:16 2013 -0700

    First assignment
\clearpage

\section{Challenges}
\begin{itemize}
\item Lseeking through the file, while reading: This caused me to run into issues of running through the file too far.
\item Not padding the spaces when pulling data from stat caused weird issues when trying to archive initially.
\end{itemize}
\clearpage

\section{Questions}
\begin{enumerate}
\item What do you think the main point of this assignment is? \\ To make us hate ourselves. No, but really, to leanr how file I/O works.
\item How did you ensure your assignment was correct? Testing details, for instance. \\ I tested it with ar. It returns the same results as ar for -q and -tv.
\item What did you learn? \\ File I/O is tough to deal with at low levels, and we should be appreciative of how high level it's become in languages like Python.
\end{enumerate}

\end{document}
