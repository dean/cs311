\documentclass[fleqn,10pt,titlepage]{article}

\usepackage{balance}
\usepackage[TABBOTCAP, tight]{subfigure}
\usepackage{enumitem}

\usepackage[top=0.8in, left=1in, bottom=0.8in, right=1in]{geometry}

\usepackage{pstricks, pst-node}

\usepackage{geometry}
\geometry{textheight=10in, textwidth=7.5in}

\newcommand{\cred}[1]{{\color{red}#1}}
\newcommand{\cblue}[1]{{\color{blue}#1}}

\usepackage{hyperref}

\def\name{Dean Johnson}

%% The following metadata will show up in the PDF properties
\hypersetup{
  colorlinks = true,
  urlcolor = black,
  pdfauthor = {\name},
  pdfkeywords = {cs311 ``operating systems'' files filesystem I/O},
  pdftitle = {CS 311 Project 2: UNIX File I/O},
  pdfsubject = {CS 311 Project 2},
  pdfpagemode = UseNone
}

\pagenumbering{gobble}
\parindent = 0.0 in
\parskip = 0.2 in

\begin{document}

\section{Design}
\hrule
\subsection{Initial Implementation}
My overall idea for an inital design involves using the struct in the ar.h file, to store all information about the archive, and then 
information on each archived file below it, following the specifications listed in ar.h.
\subsubsection{Archiving}
First write information about the files, to the file descriptor of the archive. Store this first. Next, follow the same steps on each
file provided in the arguments, by saving the file descriptor information, then the content directly afterwards.

\subsubsection{Extraction}
Read the information about the archive file, then begin extraction on the first file stored inside, and so on.
\subsection{Deviations}
Deviations from initial design...
\clearpage

\section{Work Log}
Git log...
\clearpage

\section{Challenges}
\begin{itemize}
\item Challenge 1...
\item Challenge 2...
\end{itemize}
\clearpage

\section{Questions}
\begin{enumerate}
\item What do you think the main point of this assignment is? \\ Answer
\item How did you ensure your assignment was correct? Testing details, for instance. \\ Thorough answer.
\item What did you learn? \\ Answer
\end{enumerate}

\end{document}
