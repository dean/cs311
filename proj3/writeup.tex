\documentclass[fleqn,10pt,titlepage]{article}

\usepackage{balance}
\usepackage[TABBOTCAP, tight]{subfigure}
\usepackage{enumitem}

\usepackage[top=0.8in, left=1in, bottom=0.8in, right=1in]{geometry}

\usepackage{pstricks, pst-node}

\usepackage{geometry}
\geometry{textheight=10in, textwidth=7.5in}

\newcommand{\cred}[1]{{\color{red}#1}}
\newcommand{\cblue}[1]{{\color{blue}#1}}

\usepackage{longtable, hyperref}
\newcommand{\longtableendfoot}{Please continue at the next page}

\def\name{Dean Johnson}

%% The following metadata will show up in the PDF properties
\hypersetup{
  colorlinks = true,
  urlcolor = black,
  pdfauthor = {\name},
  pdfkeywords = {cs311 ``operating systems'' files filesystem I/O},
  pdftitle = {CS 311 Project 2: UNIX File I/O},
  pdfsubject = {CS 311 Project 2},
  pdfpagemode = UseNone
}

\pagenumbering{gobble}
\parindent = 0.0 in
\parskip = 0.2 in

\begin{document}

\section{Design}
\hrule
\subsection{Initial Implementation}
\begin{enumerate}
\item Read in number of processes to run via getopt \\
\item Use a 2D array with 2 * x number of processes for reading and writing to pipes \\
\item Push words from each line of the file to each pipe until all words are gone \\
\item Exec sort on each pipe \\
\item Pop words from each pipe in alphabetical order to stdout \\
\item Record time for each of these full processes on multiple types/sizes of files \\
\end{enumerate}
\subsection{Deviations}
When writing the initial implementation, I didn't quite understand how program was supposed to work,
so it ended up missing steps such as sending signals, and waiting on child processes for things to be
done in the parent. I also didn't realize I'd have to write my own function to evaluate which word had
the lowest value (to push to stdout next), so I wrote that as well.
\clearpage

\section{Work Log}
commit 7082052648df9f029c601f61cf641e1aefc73ced
Author: Dean Johnson <deanjohnson222@gmail.com>
Date:   Wed Nov 13 22:41:03 2013 -0800

    More additions to get rid of errors.

commit ad8a077c1e1f52334de82b9c5a1af26a6631edfe
Author: Dean Johnson <deanjohnson222@gmail.com>
Date:   Wed Nov 13 20:52:48 2013 -0800

    Lots of changes.

commit 86e866b41a0c0618d1125e2701e48c822271f068
Author: Dean Johnson <deanjohnson222@gmail.com>
Date:   Tue Nov 8 01:21:12 2013 -0800

    Added initial implementation for assignment to tex document.

commit 3e2823a316d0f1a33d7817e656269f7b7dabd931
Author: Dean Johnson <deanjohnson222@gmail.com>
Date:   Wed Nov 6 18:31:05 2013 -0800

    Init proj3

commit 073f686ca43fc8a427bd8a0c69d25c964c16ec81
Author: Dean Johnson <deanjohnson222@gmail.com>
Date:   Mon Nov 4 16:14:28 2013 -0800

    Minor cleanup.
\clearpage

\section{Challenges}
\begin{itemize}
\item Understanding how file streams work.
\item Understanding how pipes work.
\item Understanding how to implement signals and send data between processees.
\item Stringing all the confusing portions fo the project together.
\end{itemize}
\clearpage

\section{Questions}
\begin{enumerate}
\item What do you think the main point of this assignment is?:  To really teach us how pipes, files, and 
asynchonous processes work. Additionally, preparing us for the next assignment using threads. (Even though
the environment is different)\\ 
\item How did you ensure your assignment was correct? Testing details, for instance. I tested my code with
different amounts of words, different orders of the words, and sending different numbers of words.\\ 
\item What did you learn? I learned how to implement pipes, and send signals to processes. I sort of understood
how forking worked with children from a parent process initially, but it didn't really become concrete until I
started implementing the spawning for the children on the assignment.\\ 
\end{enumerate}

\end{document}
