\documentclass[fleqn,10pt,titlepage]{article}

\usepackage{balance}
\usepackage[TABBOTCAP, tight]{subfigure}
\usepackage{enumitem}

\usepackage[top=0.8in, left=1in, bottom=0.8in, right=1in]{geometry}

\usepackage{pstricks, pst-node}

\usepackage{geometry}
\geometry{textheight=10in, textwidth=7.5in}

\newcommand{\cred}[1]{{\color{red}#1}}
\newcommand{\cblue}[1]{{\color{blue}#1}}

\usepackage{longtable, hyperref}
\newcommand{\longtableendfoot}{Please continue at the next page}

\def\name{Dean Johnson}

%% The following metadata will show up in the PDF properties
\hypersetup{
  colorlinks = true,
  urlcolor = black,
  pdfauthor = {\name},
  pdfkeywords = {cs311 ``operating systems'' files filesystem I/O},
  pdftitle = {CS 311 Project 2: UNIX File I/O},
  pdfsubject = {CS 311 Project 2},
  pdfpagemode = UseNone
}

\pagenumbering{gobble}
\parindent = 0.0 in
\parskip = 0.2 in

\begin{document}

\section{Design}
\hrule
\subsection{Initial Implementation}
\subsubsection{Part 1}
First, I'm going to write out the assignment without threads and processes. This
will ensure I have the basics of the program working, so I'm not having to debug which
portions of the assignment are working and not working. Next, I'll implement the threads and
processes. I will use the shared memory to share the bitmap, and when using the seve from
assignment 1, I will split the marking of each number's multiple to x sections, and have 
each thread/process be in charge of marking up to those points. For the happy numbers, I'll
calculate multiple at a time and begin printing them out.
\subsubsection{Part 2}
\subsection{Deviations}
\clearpage

\section{Work Log}
Git log
\clearpage

\section{Challenges}
\begin{itemize}
\end{itemize}
\clearpage

\section{Questions}
\begin{enumerate}
\item What do you think the main point of this assignment is? \\ 
\item How did you ensure your assignment was correct? Testing details, for instance. \\ 
\item What did you learn? \\ 
\end{enumerate}

\end{document}
