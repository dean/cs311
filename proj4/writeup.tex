\documentclass[fleqn,10pt,titlepage]{article}

\usepackage{balance}
\usepackage[TABBOTCAP, tight]{subfigure}
\usepackage{enumitem}

\usepackage[top=0.8in, left=1in, bottom=0.8in, right=1in]{geometry}

\usepackage{pstricks, pst-node}

\usepackage{geometry}
\geometry{textheight=10in, textwidth=7.5in}

\newcommand{\cred}[1]{{\color{red}#1}}
\newcommand{\cblue}[1]{{\color{blue}#1}}

\usepackage{longtable, hyperref}
\newcommand{\longtableendfoot}{Please continue at the next page}

\def\name{Dean Johnson}

%% The following metadata will show up in the PDF properties
\hypersetup{
  colorlinks = true,
  urlcolor = black,
  pdfauthor = {\name},
  pdfkeywords = {cs311 ``operating systems'' files filesystem I/O},
  pdftitle = {CS 311 Project 2: UNIX File I/O},
  pdfsubject = {CS 311 Project 2},
  pdfpagemode = UseNone
}

\pagenumbering{gobble}
\parindent = 0.0 in
\parskip = 0.2 in

\begin{document}

\section{Design}
\hrule
\subsection{Initial Implementation}
\subsubsection{Part 1}
First, I'm going to write out the assignment without threads and processes. This
will ensure I have the basics of the program working, so I'm not having to debug which
portions of the assignment are working and not working. Next, I'll implement the threads and
processes. I will use the shared memory to share the bitmap, and when using the seve from
assignment 1, I will split the marking of each number's multiple to x sections, and have 
each thread/process be in charge of marking up to those points. For the happy numbers, I'll
calculate multiple at a time and begin printing them out.
\subsubsection{Part 2}
\subsection{Deviations}
When using the bitmap on the threads portion of the code, instead of assigning it a shared
memory segment, I share the pointer it is associated with to all threads. For each prime, 
I calculate if it is a happy number in each thread, then print out the number if it is indeed
happy. After that, the thread gets the next prime in the list and begins working with it.
\clearpage

\section{Work Log}
commit 444cb2906f867f70453fe0363171b81266c4ed94
Author: Dean Johnson <deanjohnson222@gmail.com>
Date:   Wed Nov 27 21:19:17 2013 -0800

    Updated timing graph for multiple values.

commit 477012ddf03b2695079531a6f28435d28684a6ae
Author: Dean Johnson <deanjohnson222@gmail.com>
Date:   Wed Nov 27 20:43:24 2013 -0800

    Added a second timings file (int_max) with threads.

commit 0a783d860b7c22cca1efcfa2a449fec280b2e277
Author: Dean Johnson <deanjohnson222@gmail.com>
Date:   Wed Nov 27 20:02:28 2013 -0800

    Updated for processes.

commit 3590b40b59013d0340f91bbc3b3b59e7afb8bf5d
Merge: bce8a04 2a3545b
Author: Dean Johnson <deanjohnson222@gmail.com>
Date:   Wed Nov 27 18:59:47 2013 -0800

    Merge branch 'master' of github.com:johnsdea/cs311

    Conflicts:
        proj4/timings.txt

commit bce8a04d723a8813d2903defdacc188e7431d274
Author: Dean Johnson <deanjohnson222@gmail.com>
Date:   Wed Nov 27 18:49:54 2013 -0800

    Added list of timings.

commit 3a8c25aa13c1bc29c8eaf4b826c87b8222b50470
Author: Dean Johnson <deanjohnson222@gmail.com>
Date:   Wed Nov 27 18:49:17 2013 -0800

    Added a graph for timings of thread counts.

commit 2a3545b1777dad32f7e35c021a4f40987173d80c
Author: Dean Johnson <deanjohnson222@gmail.com>
Date:   Wed Nov 27 18:32:28 2013 -0800

    Added timings for thread execution.

commit 774e5a839747073d77c033f75fc773881f84464f
Author: Dean Johnson <deanjohnson222@gmail.com>
Date:   Wed Nov 27 13:07:24 2013 -0800

    Updated for processes (not working yet)

commit baed985c9db53fe870c341a9594577afd9041949
Author: Dean Johnson <deanjohnson222@gmail.com>
Date:   Wed Nov 27 04:31:07 2013 -0800

    Threads work with prime generation and happy nums.

commit fb5752d701837e3589eb17572071fe0eb7c320fb
Author: Dean Johnson <deanjohnson222@gmail.com>
Date:   Wed Nov 27 04:17:00 2013 -0800

    Working threads, implemented threaded happy numbers.

commit 9d336197602970a38cfee5cc93a9958580fb9a7c
Author: Dean Johnson <deanjohnson222@gmail.com>
Date:   Tue Nov 26 23:49:41 2013 -0800

    Updated marking nums. WORKS FOR INT_MAX. -- Backup

commit b48f51e46baf83cb2195cf420a76f0b6f4e1406c
Author: Dean Johnson <deanjohnson222@gmail.com>
Date:   Tue Nov 26 23:00:03 2013 -0800

    Broke everything.

commit d43565ec7a1bf334ecf6a55b3a3823b7f7c94805
Author: Dean Johnson <deanjohnson222@gmail.com>
Date:   Tue Nov 26 04:52:50 2013 -0800

    Commit before refactor

commit dfb893033144835cc6083ad569ebcb32eb805a62
Author: Dean Johnson <deanjohnson222@gmail.com>
Date:   Tue Nov 26 00:20:42 2013 -0800

    Backup of previous implementation

commit e7b3395728e744ae7e855b375b075e9576c1a980
Author: Dean Johnson <deanjohnson222@gmail.com>
Date:   Mon Nov 25 23:20:55 2013 -0800

    Got threads working. About to test on OS-Class

commit 694287e21c7b735bfdcf835286c93f6cb106179d
Author: Dean Johnson <deanjohnson222@gmail.com>
Date:   Mon Nov 25 22:20:41 2013 -0800

    Added more thread stuff in, it's freezing though.

commit 61655e18f8c6e9293c16ad240fbd7794aec33488
Author: Dean Johnson <deanjohnson222@gmail.com>
Date:   Mon Nov 25 01:35:35 2013 -0800

    Wrote some code for paralellizing.

commit 533dde5706d6c2f2d2bb3426166c5f35f8448907
Author: Dean Johnson <deanjohnson222@gmail.com>
Date:   Mon Nov 25 00:30:29 2013 -0800

    Implemented the program w/o processes and threads.

commit dbc4e2a2ae0173df01f6967d66016938c20c8386
Author: Dean Johnson <deanjohnson222@gmail.com>
Date:   Sun Nov 24 19:18:42 2013 -0800

    Updated writeup

\clearpage

\section{Challenges}
\begin{itemize}
\item Spawning threads originally gave me a lot of trouble I tried passing an array of 
objects pointer to that array, and it was causing seg-faults. I fixed this by just sending 
the address of each thread I created to pcreate t.
\item Which method to use to count primes via threads. The biggest challenge I ran in to was
implementing the threads in a way that they didn't interact with the same bit at the same time.
My first attempt was giving each thread a number to work on, then getting the next number
(without keeping track of other threads numbers), but that ended up giving me skewed results.
The second attempt I made tried letting each thread handle a certain range in the bitmap, I'm
not quite sure what went wrong with this one, but it didn't work. Lastly, I went back to the
first attempt but kept track of other thread's numbers. This allowed me to only grab primes
that were not being iterated over currently.
\item Another big issue I had was iterating using an unsigned int, instead of a long. Even
though I was making sure the unsigned int held a number no larger than UINT MAX-1, I was 
still running into issues with it calculating correctly when adding values.
\end{itemize}
\clearpage

\section{Questions}
\begin{enumerate}
\item What do you think the main point of this assignment is? \\ I think the main part of this
assignment was to teach us threads, but to also help us contrast threads and processes. Threads
seem to have a much easier time sharing memory, and are easier to understand conceptually for me.
\item How did you ensure your assignment was correct? Testing details, for instance. \\ I 
tested with 255 for the max number initially, so when I was printing out, I could see exactly
which threads were doing certain things. Secondly, I made a function that counted the number of
unmarked primes, then compared that to Wolfram Alpha's results for the number of primes below a
given number. To make sure it was utilizing all threads, I used a lot of print statements to
print which threads were marking each number as well.
\item What did you learn? \\ I learned a lot about threads and how to manage data with them.
I found it much more conceptually understanding than processes and easier to implement threads
too. It was so easy to understand, once I had primes implemented, it took me only 20
minutes to get threads working on calculating happy primes.
\end{enumerate}

\end{document}
